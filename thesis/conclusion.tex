\startconclusionpage

Был разработан алгоритм сжатия коротких текстовых сообщений. Он основан на алгоритме Хаффмана. Был детально рассмотрен вопрос о размере получаемого словаря и то, как кодировать слова, которые в него не попали. Разработанный метод был детально описан.

Разработанный алгоритм был применен для сжатия сообщений в ООО <<В Контакте>>, что позволило сократить их размер на 5-7\% по сравнению с предыдущей версией сжатия. Такое улучшение в масштабах <<В Контакте>> соответсвует экономии нескольких терабайт оперативной памяти.

Разработанный алгоритм сложно сравнивать с существующими алгоритмами сжатия, так как они не расчитаны на сжатие коротких сообщений. Все найденные решения проигрывают разработанному алгоритму по степени сжатия.

Разработанный алгоритм, а также все дополнительные материалы, доступны в \cite{github}.

Как одно из направлений развития алгоритма можно попробовать использовать арифметическое кодирование вместо алгоритма Хаффмана.